\documentclass{beamer}
\usepackage{lmodern}
\usepackage{wasysym}
\usepackage{graphicx}
\usepackage{listings}
\usepackage[T1]{fontenc}
\usepackage[utf8]{inputenc}
\usepackage{dirtree}
\date{\today}
\lstdefinestyle{basic}{
		captionpos=t,%
		basicstyle=\footnotesize\ttfamily,%
		numberstyle=\tiny,%
		numbers=left,%
		stepnumber=1,%
		frame=single,%
		showspaces=false,%
		showstringspaces=false,%
		showtabs=false,%
		%
		keywordstyle=\color{blue},%
		identifierstyle=,%
		commentstyle=\color{gray},%
		stringstyle=\color{magenta}%
}
\usecolortheme{default}
\setbeamertemplate{navigation symbols}{}
\setbeamertemplate{blocks}[rounded][shadow=true]
\setbeamercolor{block title}{bg=blue!20!white}
\setbeamercolor{block body}{bg=blue!5!white}

\begin{document}

\input{cc_beamer.tex}
\begin{frame}
		\title{Git-Workshop}
		\author{Moritz Grimm}
		\date{}
		\titlepage
		\begin{center}
			\CcGroupByNcSa{0.83}{0.95ex}\\
			\vspace{0.4cm}
			\tiny{Veröffentlicht unter der CreativeCommons-Lizenz (By, Nc, Sa)}\\
			\vspace{0.2cm}
			\tiny{Basierend auf Material von Valentin Haenel (\url{https://github.com/esc/git-workshop})}\\
			\vspace{0.2cm}
			\tiny{Mit Anpassungen von Julius Plenz (\url{https://github.com/Feh/git-workshop})}\\
			\vspace{0.2cm}
			\tiny{und Felix Bürkle (\url{https://github.com/NiciDieNase/git-workshop})}
		\end{center}
\end{frame}
\begin{frame}
	\includegraphics[height=\textheight]{bilder/yawning_lion.jpg}
	Photo by Keven Law
	(66164549@N00 @ flickr.com)
\end{frame}
\begin{frame}
 \frametitle{Der Plan}

 \begin{itemize}
	\item Grundlegende Arbeitsschritte in Git
	\item Das Objektmodell -- eine theoretische Grundlage
	\item Parallele Entwicklung: Branches und Merges
	\item Entwicklung koordinieren: Ein Branching-Modell
	\item Die Geschichte umschreiben: Rebase
	\item Verteiltes Git: Commits hoch- und runterladen
	\item Verschiedene Workflows
\end{itemize}

 \end{frame}
\begin{frame}
	\includegraphics{bilder/yawning_leopard.jpg}\\
	by Art G. (digitalart @ flickr.com)
\end{frame}
\begin{frame}
 \frametitle{Motivation: Warum Versionskontrolle?}

 \begin{itemize}
	\item \textbf{Sicherheit:} Versionskontrolle schützt vor Verlusten
	\item \textbf{Dokumentation:} Wer hat wann was gemacht?
	\item \textbf{Fokussierung:} Entwicklung logisch gliedern
	\item \textbf{Kollaboration:} Mit anderen Leuten an den gleichen Dateien arbeiten
	\item \textbf{Partizipation:} Jeder kann mitmachen (GitHub etc.)
\end{itemize}

 \end{frame}
\begin{frame}
 \frametitle{Wer bin ich? -- Name und E-Mail einstellen}



\begin{itemize}
	\item Für alle Projekte (wird in \texttt{\~{}/.gitconfig} gespeichert)
\begin{itemize}
	\item \texttt{git config -{}-global user.name "Max Mustermann"}
	\item \texttt{git config -{}-global user.email max@mustermann.de}
\end{itemize}
\end{itemize}

\begin{itemize}
	\item ... oder alternativ nur für das aktuelle Projekt:
\begin{itemize}
	\item \texttt{git config user.email maintainer@cool-project.org}
\end{itemize}
\end{itemize}

\begin{itemize}
	\item Außerdem, für die, die wollen: Farbe!
\begin{itemize}
	\item \texttt{git config -{}-global color.ui auto}
\end{itemize}
\end{itemize}


 \end{frame}
\begin{frame}
 \frametitle{Ein Projekt importieren oder erstellen}

\begin{itemize}
	\item Ein neues Projekt erstellt man wie folgt:
\begin{itemize}
	\item \texttt{mkdir \emph{projekt}}
	\item \texttt{cd \emph{projekt}}
	\item \texttt{git init}
\end{itemize}
\end{itemize}

\begin{itemize}
	\item Um ein bestehendes Projekt zu importieren, »klont« man es mit seiner gesamten Versionsgeschichte:
\begin{itemize}
	\item \texttt{git clone git@github.com:fsi-hska/git-workshop.git}
\end{itemize}
\end{itemize}


 \end{frame}
\begin{frame}
 \frametitle{Begriffsbildung - Index}



\begin{itemize}
	\item \textbf{Index}/\textbf{Staging Area}: Bereich zwischen dem Arbeitsverzeichnis und dem Repository, in die Änderungen für den nächsten \textbf{Commit} gesammelt werden
\end{itemize}


 \end{frame}
\begin{frame}
 \frametitle{Begriffsbildung - Commit}



\begin{itemize}
	\item \textbf{Commit}: Eine Änderung an einer oder mehrerer Dateien, versehen mit Metadaten wie Autor, Datum und Beschreibung
\end{itemize}


 \end{frame}
\begin{frame}
 \frametitle{Begriffsbildung - Repository}



\begin{itemize}
	\item \textbf{Repository}: Eine Datenbank für Commits, dort wird die Versionsgeschichte aufgezeichnet
\end{itemize}


 \end{frame}
\begin{frame}
 \frametitle{Begriffsbildung - Referenz}



\begin{itemize}
	\item \textbf{Referenz}: Jeder \textbf{Commit} wird durch eine eindeutige SHA1-Summe identifiziert. Eine Referenz »zeigt« auf einen bestimmten Commit
\end{itemize}


 \end{frame}
\begin{frame}
 \frametitle{Begriffsbildung - Branch}



\begin{itemize}
	\item \textbf{Branch}: Ein »Zweig«, eine Abzweigung im Entwicklungszyklus, z.\,B. um ein neues Feature einzuführen.
\end{itemize}


 \end{frame}
\begin{frame}
 \frametitle{Ein typischer Arbeitsablauf}



\begin{itemize}
	\item Eine \texttt{\emph{datei}} verändern, und die Änderungen in das Repository »einchecken«:
\end{itemize}
\begin{enumerate}
	\item \texttt{\$EDITOR \emph{datei}}
	\item \texttt{git status}
	\item \texttt{git add \emph{datei}}
	\item \texttt{git commit -m '\emph{datei} angepasst'}
	\item \texttt{git show}
\end{enumerate}


 \end{frame}
\begin{frame}
 \frametitle{Index / Staging Area}



\begin{itemize}
	\item Im \emph{Index} bzw. der \emph{Staging-Area} werden Veränderungen für den nächsten Commit vorgemerkt
	\item So kann der Inhalt von einem Commit schrittweise aus einzelnen Veränderungen zusammengestellt werden
\end{itemize}


 \end{frame}
\begin{frame}
 \frametitle{Ausgangsstellung}



\begin{itemize}
	\item Alle auf dem gleichen Stand
\end{itemize}

\begin{center}
\includegraphics[ scale=0.25]{bilder/index-stage1.pdf}
\end{center}


 \end{frame}
\begin{frame}
 \frametitle{Veränderungen machen}



\begin{itemize}
	\item Veränderungen werden im Working-Tree gemacht
\end{itemize}

\begin{center}
\includegraphics[ scale=0.25]{bilder/index-stage2.pdf}
\end{center}


 \end{frame}
\begin{frame}
 \frametitle{Dem Index hinzufügen -- \texttt{git add}}



\begin{itemize}
	\item Die Veränderungen im Working-Tree $\rightarrow$ Index
\end{itemize}

\begin{center}
\includegraphics[ scale=0.25]{bilder/index-stage3.pdf}
\end{center}


 \end{frame}
\begin{frame}
 \frametitle{Einen Commit erzeugen -- \texttt{git commit}}



\begin{itemize}
	\item Alle Veränderungen im Index $\rightarrow$ Commit
\end{itemize}

\begin{center}
\includegraphics[ scale=0.25]{bilder/index-stage4.pdf}
\end{center}


 \end{frame}
\begin{frame}
 \frametitle{Resultat}



\begin{itemize}
	\item Alle wieder	auf dem gleichen Stand
\end{itemize}

\begin{center}
\includegraphics[ scale=0.25]{bilder/index-stage5.pdf}
\end{center}


 \end{frame}
\begin{frame}
 \frametitle{Referenzen und ignorierte Dateien}



Relative Referenzen:
\begin{itemize}
	\item \texttt{HEAD}: Der letzte Commit (wird per \texttt{git show} angezeigt)
	\item \texttt{HEAD$^\mathbf{\wedge}$}: Der vorletzte Commit
	\item \texttt{HEAD\textasciitilde{}\emph{N}}: Der \emph{N}.-letzte Commit
\end{itemize}


 \end{frame}
\begin{frame}
 \frametitle{Informationen über das Repository erhalten}



\begin{itemize}

	\item[] Den jüngsten Commit im vollen Umfang anschauen:
	\begin{itemize}
		\item \texttt{git show}
	\end{itemize}

	\item[] Die gesamte Versionsgeschichte anzeigen:
	\begin{itemize}
		\item \texttt{git log}
	\end{itemize}

	\item[]  Was hat sich verändert?
	\begin{itemize}
		\item \texttt{git diff}
	\end{itemize}

	\item[] Das Repository visualisieren:
	\begin{itemize}
		\item \texttt{gitk}
		\item \texttt{gitg}
	\end{itemize}

	\item[] ... oder textbasiert:
	\begin{itemize}
		\item \texttt{tig}
	\end{itemize}

\end{itemize}


 \end{frame}
\begin{frame}
 \frametitle{Änderungen rückgängig machen}



Einen neuen Commit erstellen, der eine alte Änderung rückgängig macht:
\begin{itemize}
	\item \texttt{git revert \emph{commit}}
\end{itemize}

Den Index zurücksetzen:
\begin{itemize}
	\item \texttt{git reset HEAD}
\end{itemize}

Den Zustand von vor zwei Commits wiederherstellen:
\begin{itemize}
	\item \texttt{git checkout HEAD\textasciitilde{}2}
\end{itemize}

Die letzten zwei Commits \textcolor{red}{\textbf{unwiederbringlich}} löschen:
\begin{itemize}
	\item \texttt{git reset -{}-hard HEAD\textasciitilde{}2}
\end{itemize}


 \end{frame}
\begin{frame}
 \frametitle{Branches: Abzweigungen}



Wir arbeiten schon die ganze Zeit im \texttt{master}-Branch!

\vspace{.5cm}


Was genau sind Branches? -- Nichts anderes als Referenzen auf den jeweils
obersten Commit einer Versionsgeschichte.

\vspace{.5cm}


Branches ...
\begin{itemize}
	\item erstellen: \texttt{git branch \emph{name}}
	\item auschecken: \texttt{git checkout \emph{name}}
	\item erstellen und direkt auschecken: \texttt{git checkout -b \emph{name}}
	\item auflisten: \texttt{git branch -v}
	\item löschen: \texttt{git branch -d \emph{name}}
\end{itemize}


\vspace{.5cm}

Idealisierter Workflow: Ein Branch pro neuem Feature oder Bugfix.


 \end{frame}
\begin{frame}
 \frametitle{Beispiel: Zwei Branches}



Zwei Branches erstellen, und auf jedem einen Commit machen. Dann das
Resultat in \texttt{gitk} anschauen.

\begin{itemize}
	\item \texttt{git branch eins}
	\item \texttt{git checkout eins}
	\item Commit machen
	\item \texttt{git checkout master}
	\item \texttt{git checkout -b zwei}
	\item Commit machen
	\item \texttt{gitk -{}-all}
\end{itemize}


 \end{frame}
\begin{frame}
 \frametitle{Beispielprojekt: Was wollen wir speichern}



Angenommen, wir wollen folgendes Verzeichnis speichern:


\vspace{0.5cm}


\dirtree{.1 /.
.2 hello.py.
.2 README.
.2 test/.
.3 test.sh.
}


 \end{frame}
\begin{frame}
 \frametitle{Objektmodell}



\begin{itemize}
	\item \emph{Blob}: Enthält den Inhalt einer Datei
	\item \emph{Tree}: Eine Sammlung von Tree- und Blob-Objekten
	\item \emph{Commit}: Besteht aus einer Referenz auf einen Tree mit zusätzlichen Informationen
\begin{itemize}
	\item \emph{Author} und \emph{Commiter}
	\item \emph{Parents}
	\item \emph{Commit-Message}
\end{itemize}
\end{itemize}

\begin{figure}[ht]
\includegraphics[ scale=0.6]{bilder/objekte.pdf}
\end{figure}


 \end{frame}
\begin{frame}
 \frametitle{SHA-1 IDs}



\begin{itemize}
	\item Objekte werden mit \emph{SHA-1 IDs} identifiziert
	\item Dies ist der \emph{Objekt-Name}
	\item Wird aus dem Inhalt berechnet
	\item \emph{SHA-1} ist eine sogenannte Hash-Funktion; sie liefert für eine Bit-Sequenz mit der maximalen Länge von $2^{64}-1$ Bit ($\approx$2 Exbibyte) in eine Hexadezimal-Zahl der Länge 40 (d.\,h.\ 160 Bits)
	\item Die resultierende Zahl ist eine von $2^{160} (\approx 1.5\cdot 10^{49})$ möglichen Zahlen und ziemlich einzigartig
\end{itemize}

\begin{figure}[htp]
		\includegraphics[ scale=0.6]{bilder/sha.pdf}
\end{figure}


 \end{frame}
\begin{frame}
 \frametitle{Objektverwaltung}



\begin{itemize}
	\item Alle Objekte werden von Git in der \emph{Objektdatenbank} (genannt Repository) gespeichert
	\item Die Objekte sind durch ihre SHA-1 ID eindeutig adressierbar
\end{itemize}

\begin{itemize}
	\item Für jede Datei erzeugt Git ein Blob-Objekt
	\item Für jedes Verzeichnis erzeugt Git ein Tree-Objekt
	\item Ein Tree-Objekt enthält die Referenzen (SHA1 IDs) auf die in dem Verzeichnis enthaltenen Dateien
\end{itemize}


 \end{frame}
\begin{frame}
 \frametitle{Zusammenfassung}



Ein Git-Repository enthält Commits; diese wiederum referenzieren Trees und Blobs, sowie ihren direkten Vorgänger

\begin{figure}[ht]
		\includegraphics[ scale=0.6]{bilder/objekte-zusammenhang.pdf}
\end{figure}


 \end{frame}
\begin{frame}
 \frametitle{Commit Graph}



Ein Repository ist ein \emph{Gerichteter Azyklischer Graph}\\
Engl.: Directed Acyclic Graph (DAG)

\begin{center}
\includegraphics[ scale=0.14]{bilder/graph.pdf}
\end{center}


 \end{frame}
\begin{frame}
 \frametitle{Branches und Tags}



Branches und Tags sind Zeiger auf Knoten in dem Graphen.\\
\textcolor{white}{Engl.}

\begin{center}
\includegraphics[ scale=0.14]{bilder/graph-mit-refs.pdf}
\end{center}


 \end{frame}
\begin{frame}
 \frametitle{Graph-Struktur}



\begin{itemize}
	\item Die gerichtete Graph-Struktur entsteht, da in jedem Commit Referenzen auf direkte Vorfahren gespeichert sind
\end{itemize}

\begin{itemize}
	\item Integrität kryptographisch gesichert
\end{itemize}

\begin{itemize}
	\item Git-Kommandos manipulieren die Graph-Struktur
\end{itemize}


 \end{frame}
\begin{frame}
 \frametitle{Merging: Branches Zusammenfügen}



Simple Merge:
\begin{itemize}
	\item \texttt{git merge \emph{neues-feature}}
\end{itemize}


\vspace{.5cm}


Fast-Forward Merge:
\begin{itemize}
	\item Wird \texttt{\emph{topic}} in \texttt{\emph{master}} gemerget und \texttt{\emph{topic}} basiert auf \texttt{\emph{master}}, dann wird kein Merge-Commit erstellt, sondern nur der Zeiger »weitergerückt« bzw. »vorgespult«.
\end{itemize}


 \end{frame}
\begin{frame}
 \frametitle{Vor dem Merge}



\begin{center}
\includegraphics[scale=0.12]{bilder/merge-vorher.pdf}
\end{center}

\begin{itemize}
	\item \texttt{topic} ist fertig und soll in \texttt{master} integriert werden
\end{itemize}


 \end{frame}
\begin{frame}
 \frametitle{Nach dem Merge}



\begin{center}
\includegraphics[scale=0.12]{bilder/merge-nachher.pdf}
\end{center}

\begin{itemize}
	\item Im \texttt{master} ausführen: \texttt{git merge topic}
\end{itemize}


 \end{frame}
\begin{frame}
 \frametitle{Vor dem Fast-Forward}



\begin{center}
\includegraphics[scale=0.12]{bilder/ff-vorher.pdf}
\end{center}

\begin{itemize}
	\item In \texttt{master} hat sich nichts getan, \texttt{topic} ist fertig
\end{itemize}


 \end{frame}
\begin{frame}
 \frametitle{Nach dem Fast-Forward}



\begin{center}
\includegraphics[scale=0.12]{bilder/ff-nachher.pdf}
\end{center}

\begin{itemize}
	\item \texttt{master} wird »weitergerückt«, bzw. »vorgespult«
\end{itemize}


 \end{frame}
\begin{frame}
 \frametitle{Hilfe, Konflikte!}



Bei einem \texttt{merge} kann es zu Konflikten kommen. Wie geht man damit um?
\begin{itemize}
	\item \texttt{\$EDITOR \emph{konfliktdateien}}
	\item \texttt{git add \emph{konfliktdateien}}
	\item \texttt{git commit -m "\emph{Merge-Konflikt behoben}"}
\end{itemize}


\vspace{.5em}

Das Unterfangen abbrechen:
\begin{itemize}
	\item \texttt{git reset -{}-hard HEAD}
\end{itemize}


 \end{frame}

\begin{frame}
 \frametitle{Einen Commit ändern}



\begin{itemize}
	\item Commit ändern = Neuen Commit erstellen, alten wegschmeißen
	\item Den letzten Commit (\texttt{HEAD}) ändern:
\end{itemize}

\begin{enumerate}
	\item \texttt{\$EDITOR \emph{datei}}
	\item \texttt{git add \emph{datei}}
	\item \texttt{git commit -{}-amend}
\end{enumerate}

\begin{itemize}
	\item Tiefer liegende Commits (\texttt{HEAD\textasciitilde{}1} etc.) können so nicht geändert werden!
\end{itemize}


 \end{frame}
\begin{frame}
 \frametitle{Vor dem Rebase}



\begin{center}
\includegraphics[scale=0.12]{bilder/rebase-vorher.pdf}
\end{center}

\begin{itemize}
	\item \texttt{\emph{topic}} soll auf der neusten Version von \texttt{master} basieren
\end{itemize}


 \end{frame}
\begin{frame}
 \frametitle{Nach dem Rebase}



\begin{center}
\includegraphics[scale=0.12]{bilder/rebase-nachher.pdf}
\end{center}

\begin{itemize}
	\item \texttt{git rebase master \emph{topic}}
\end{itemize}


 \end{frame}
\begin{frame}
 \frametitle{Rebase: Auf eine neue Basis bauen}



\begin{itemize}
	\item \textbf{Rebase}: Einen Branch auf eine »neue Basis« stellen.
\end{itemize}
\begin{block}{\texttt{master} als neue Basis für \texttt{\emph{topic}}}
\texttt{git checkout \emph{topic}} \\
\texttt{git rebase master}
\end{block}

\begin{block}{Alternativ}
\texttt{git rebase master \emph{topic}}
\end{block}


 \end{frame}
\begin{frame}
 \frametitle{Rebasing: eine Warnung}



\begin{itemize}
	\item \textcolor{red}{\textbf{Wichtig:}} Man darf \textbf{\emph{niemals}} Commits aus einem bereits veröffentlichten Branch -- auf dem also womöglich Andere ihre Arbeit basieren -- durch \texttt{git rebase} verändern!
	\item \textbf{Daher: Nur Unveröffentlichtes gegen Veröffentlichtes rebasen:}
\begin{itemize}
	\item \texttt{git rebase origin/master}
	\item \texttt{git rebase v1.1.23}
\end{itemize}
\end{itemize}


 \end{frame}

\begin{frame}
 \frametitle{Hinaus in die weite Welt!}
\begin{itemize}
	\item Wir wollen unsere Arbeit mit der anderer Entwickler austauschen!
	\item Durch die verteilte Architektur von \texttt{git} braucht es keinen \emph{zentralen} Server zu geben.
	\item Das Entwicklerteam muss sich auf einen \emph{Workflow} einigen:
	\begin{itemize}
		\item Shared Repository
		\item Maintainer/Blessed Repository
		\item Patch-Queue per E-Mail
		\item ... oder auch alles durcheinandergemixt.
	\end{itemize}
\end{itemize}
 \end{frame}

\begin{frame}
 \frametitle{Zentralisiert}
\begin{figure}[ht]
		\includegraphics[ scale=0.35]{bilder/centralised.pdf}
\end{figure}

\begin{itemize}
	\item Ein einziges zentrales Repository
	\item Alle Entwickler haben Schreibzugriff
\end{itemize}


 \end{frame}
\begin{frame}
 \frametitle{Öffentliche Entwickler-Repositories}



\begin{figure}[ht]
		\includegraphics[ scale=0.35]{bilder/developer-public.pdf}
\end{figure}

\begin{itemize}
	\item Ein öffentliches Repository pro Entwickler
	\item Der Projektleiter integriert Verbesserungen
\end{itemize}


 \end{frame}
\begin{frame}
 \frametitle{Patch-Queue per Email}



\begin{figure}[ht]
		\includegraphics[ scale=0.32]{bilder/patches-per-mail.pdf}
\end{figure}

\begin{itemize}
	\item Stark vom Kernel und Git selbst verwendet
\end{itemize}


 \end{frame}
\begin{frame}
 \frametitle{Remote Repositories / Remote Branches}



Remote Repositories verwalten:
\begin{itemize}
	\item \texttt{git remote -v}
	\item \texttt{git remote add \emph{name} \emph{url}}
	\item \texttt{git remote rm \emph{name}}
	\item \texttt{git remote update}
\begin{itemize}
	\item Fragt bei allen Remote Repositories an, ob es neue Commits gibt. (Eigene Commits werden durch dieses Kommando \textbf{nicht} veröffentlicht!)
\end{itemize}
\end{itemize}

Details der Repositories ändern (z.\,B. Vertipper):
\begin{itemize}
	\item \texttt{\$EDITOR .git/config}
\end{itemize}

Remote Branches auflisten:
\begin{itemize}
	\item \texttt{git branch -r}
\end{itemize}


 \end{frame}
\begin{frame}
 \frametitle{Remote Branches vs. Remote Tracking Branches}



\includegraphics[width=\textwidth]{bilder/clone.pdf}


 \end{frame}
\begin{frame}
 \frametitle{Fremden Code holen, eigenen versenden}



Aus einem anderen Repository neuen Code »ziehen«:
\begin{itemize}
	\item \texttt{git pull \emph{remote} \emph{branch}}
\begin{itemize}
	\item \texttt{git pull blessed master}
\end{itemize}
\end{itemize}
Was hinter den Kulissen passiert:
\begin{enumerate}
	\item \texttt{git fetch \emph{remote} \emph{branch}}
	\item \texttt{git merge \emph{remote}/\emph{branch}}
\end{enumerate}


\vspace{.5cm}

Eigene Commits »pushen« oder per E-Mail senden:
\begin{itemize}
	\item \texttt{git push \emph{remote} \emph{branch}}
	\item \texttt{git format-patch \emph{seit-wann}}
\end{itemize}


 \end{frame}
\begin{frame}
 \frametitle{Konventionen}



\begin{itemize}
	\item Wiederholter Einsatz von \texttt{git pull} erzeugt viele unnötige Merges
	\item Konvention:
\begin{itemize}
	\item Nicht im \texttt{master} entwickeln
	\item \texttt{git remote update}, \texttt{master} \emph{immer} Fast-Forwarden
	\item Eigene Branches per \texttt{merge} in \texttt{master} integrieren
\end{itemize}
\end{itemize}

\begin{block}{FF-Merge erzwingen}
\texttt{git merge -{}-ff-only origin/master}\\
\texttt{git config -{}-global alias.fm 'merge -{}-ff-only'}
\end{block}


 \end{frame}

%\begin{frame}
% \frametitle{Let's develop!}
%\begin{itemize}
%	\item Wir entwickeln in Plaintext -- Markdown macht's schön!
%\end{itemize}
%\begin{itemize}
%	\item Eine Hand voll Teams (5?)
%	\item Jedes Team wählt einen Koordinator
%	\item Koordinatoren arbeiten dem Maintainer zu
%	\item Entwickler: Tauscht euch aus! (Remotes, Patches, \ldots{})
%\end{itemize}
% \end{frame}

%\begin{frame}
% \frametitle{Die Infrastruktur}
%\begin{figure}
%\includegraphics[ scale=0.6]{infrastruktur-crop.pdf}
%\end{figure}
% \end{frame}
%
%\begin{frame}
% \frametitle{Wir tauschen uns aus}
%\begin{itemize}
%	\item Remote \emph{blessed} klonen (read-only):
%\begin{itemize}
%	\item \texttt{git clone -o blessed \hspace*{.4cm} git@github.com:NiciDieNase/workshop-uebung.git}
%	\item \texttt{cd workshop-uebung}
%\end{itemize}
%\end{itemize}
%\vspace{.2cm}
%\begin{itemize}
%	\item Schreibbares Repository eintragen: \emph{meins}
%\begin{itemize}
%	\item \parbox[t]{9cm}{\texttt{git remote add meins}\\\texttt{\hspace*{.5cm}ssh://git@github.com:\textit{username}/workshop-uebung.git/}}
%\end{itemize}
%\end{itemize}
%\vspace{.2cm}
%\begin{itemize}
%	\item Repository eines anderen Teilnehmers hinzufügen:
%\begin{itemize}
%	\item \texttt{git remote add \textit{username} \hspace*{.4cm} git://git@github.com:\textit{username}/workshop-uebung.git/}
%\end{itemize}
%\end{itemize}
%\vspace{.2cm}
%\begin{itemize}
%	\item Schauen, »wie weit« die anderen sind:
%\begin{itemize}
%	\item \texttt{git remote update}
%	\item \texttt{gitk -{}-all}
%\end{itemize}
%\end{itemize}
%\end{frame}

\begin{frame}
 \frametitle{Kür: Was noch fehlt}
\begin{itemize}
	\item Rebase
	\item \texttt{git stash}
	\item Remote Branches löschen
	\item Git-Aliase
	\item Tags
	\item Reflog
\end{itemize}


 \end{frame}

\frame{Bonus-Folien}

\begin{frame}
 \frametitle{Rebase Interaktiv}
\begin{itemize}
	\item Das ist \emph{Advanced Git Magic} -- und will geübt sein!
	\item Rebase-Prozess anhalten, Commits »mittendrin« ändern, weiterlaufen lassen
\end{itemize}

\begin{block}{Interaktives Rebase}
\texttt{git rebase -i master \emph{topic}}
\end{block}

\begin{itemize}
	\item Anwendungsfälle \emph{nur} lokal und für die \emph{eigenen} Commits
\begin{itemize}
	\item Patch-Serie neu strukturieren
	\item Typos aus den eigenen Commits entfernen
	\item Offensichtliche Fehler glattbügeln
\end{itemize}
\end{itemize}
 \end{frame}

\begin{frame}
 \frametitle{Rebase Interaktiv: Beispiele}



\begin{block}{Zwei Commits zusammenfassen}
\texttt{git rebase -i HEAD\textasciitilde{}\emph{n}}\\
$\rightarrow$ \texttt{pick} des zweiten Commits durch \texttt{fixup} ersetzen
\end{block}

\begin{itemize}
	\item Einen Commit verschieben: Die Zeilen vertauschen
	\item Einen Commit editieren: mit \texttt{edit} markieren
	\item Einen Commit aufteilen: Siehe Cheatsheet
\end{itemize}
 \end{frame}

\begin{frame}
 \frametitle{Whitespace und EOL}

\begin{itemize}
	\item Was ist kaputter Whitespace?
	\item \texttt{git diff -{}-check} (z.\,B. $\rightarrow$ Hook)
\end{itemize}

\begin{itemize}
	\item \emph{Zeilenende:} Windows (CRLF) vs. UNIX (LF)
	\item \texttt{core.eol} bestimmt, was zu tun ist: \texttt{lf}, \texttt{crlf} oder \texttt{native}
	\item Git-Attribut \texttt{text} für Dateien, die automatisch konvertiert werden sollen
\begin{itemize}
	\item \texttt{echo '*.c text' > .gitattributes}
\end{itemize}
	\item \texttt{core.safecrlf}: Konvertierung verbieten, wenn ein Mix aus CRLF und LF vorhanden ist
	\item Mehr Infos: \texttt{gitattributes(5)}
\end{itemize}
 \end{frame}

\end{document}

